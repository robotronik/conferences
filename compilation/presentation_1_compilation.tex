
\section{Qu'est-ce que la compilation ?}

\subsection{Parler avec le processeur : le jeu d'instructions}

\begin{frame}[fragile]
\frametitle{\insertsubsection}
\begin{itemize}
  \item Ensemble des instructions exécutables par un processeur
    \begin{itemize}
    \item \texttt{ADD}, \texttt{AND}, \texttt{OR}, \texttt{CMP} (Compare),\…
     \item Division flottante (\texttt{FDIV}), Sinus/Cosinus (\texttt{FSIN, FCOS, FSINCOS}),\…
     \item Accès aux périphériques, changement de fréquence du processeur,\…
    \end{itemize}
  \item Le language Assembleur permet de représenter un code binaire:
    \begin{itemize}
      \item Code binaire : 10110000 01100001
      \item Code assembleur : \texttt{movb \$0x61,\%al}\\
            10110000 = \texttt{movb \%al}\\
            01100001 = \texttt{\$0x61}
      \item Action : écrire le nombre 0x61 = 97 dans le registre al.
    \end{itemize}
\end{itemize}
\end{frame}

\subsection{Du code source au binaire}
\subsubsection{Préprocesseur}
\begin{frame}[fragile]
\begin{itemize}
  \item Une étape de transformation \textit{statique} du code source.
  \item Cherche les fichiers inclus par : \texttt{\#include "mon\_header.h"}
  \item \textbf{Remplace} la ligne du \texttt{\#include} par le \textit{contenu} du fichier inclus
  \item Insère les macros là où elles sont appelées
\begin{tabular}{p{5cm} p{5cm}}
\begin{lstlisting}
#include <stdio.h>
#define HELLO_FR "Bonjour !"
#define CARRE(x) x*x

int main()
{
  printf(HELLO_FR "\n");
  printf("5^2 = %d", CARRE(5));
}
\end{lstlisting} & 
\begin{lstlisting}
<contenu de stdio.h recopié ici>



int main()
{
  printf("Bonjour !" "\n");
  printf("5^2 = %d", 5*5);
}
\end{lstlisting}
\end{tabular}
\end{itemize}
\end{frame}

\subsubsection{Les dangers du préprocesseur}
\begin{frame}[fragile]
\begin{center}
\begin{tabular}{p{5.5cm} p{5.5cm}}
\begin{lstlisting}
#include <stdio.h>
#define OK printf("valeur valide");
#define NOK printf("valeur erronée");

if (is_valid(n))
  OK;
else
  NOK;
\end{lstlisting} & 
\begin{lstlisting}
<contenu de stdio.h recopié ici>



int main()
{
  printf("Bonjour !" "\n");
  printf("5^2 = %d", 5*5);
}
\end{lstlisting} \\
\begin{lstlisting}
#define CIRCONFERENCE(x,y) x + x + y + y
printf("demi-circonférence: %d", CIRCONFERENCE(10, 10) / 2);
\end{lstlisting}
\end{tabular}
\end{center}
\end{frame}
\subsubsection{Compilation}
\begin{frame} auie \end{frame}
\subsubsection{Link}
\begin{frame} auie \end{frame}
