
\section{Qu'est-ce que la compilation ?}

\subsection{Comment qu'on exécute du code ?}
\begin{frame}[fragile]
\frametitle{\insertsubsection}
\begin{block}{Du code source au résultat}

  \begin{tikzpicture}[main node/.style={rectangle,draw,minimum size=0.5cm,align=left},]
        
    \node[main node, label=\textbf{Source :} \texttt{main.c}] (1) {
      \texttt{int main()\{}\\
      \texttt{~~~~printf("Hello");}\\
      \texttt{~~~~\…}\\
      \texttt{\}}
    };
    \node[main node] (2) [right = 1cm of 1] {Compilation};
    \node[main node, label=Code objet] (3) [right = 1cm of 2] {
      \texttt{01001001}\\
      \texttt{10010100}\\
      \texttt{11000011}\\
      \texttt{10011011}
    };
    \node[main node, label=Résultat de l'exécution] (4) [right = 1cm of 3] {
      \texttt{Hello}\\
      \texttt{The result is 42}
    };

    \draw[->] (1) edge (2) (2) edge (3) (3) edge (4);
  \end{tikzpicture}
\end{block}

\begin{block}{Les étapes de la compilation}
  \begin{tikzpicture}[main node/.style={rectangle,draw,minimum size=0cm,align=left},]
        
    \node[main node, label=Fichier \texttt{.c}] (1) {Code Source};
    \node[main node, label=Préprocesseur] (2) [right = 0.4cm of 1] {
      Code reformaté:\\
      Recherche des includes\\
      Résolution des macros
    };
    \node[main node, label=Compilateur] (3) [right = 0.4cm of 2] {
      Code objet\\
      Optimisation
    };
    \node[main node, label=Linker (Éditeur de liens)] (4) [right = 0.3cm of 3] {
      Recherche des librairies\\
      Résolution des symboles\\
      Génération de l'exécutable/librarie
    };

    \draw[->] (1) edge (2) (2) edge (3) (3) edge (4);
  \end{tikzpicture}
\end{block}
\end{frame}


\subsection{Parler avec le processeur : le jeu d'instructions}

\begin{frame}[fragile]
\frametitle{\insertsubsection}
\begin{itemize}
  \item Ensemble des instructions exécutables par un processeur
    \begin{itemize}
    \item \texttt{ADD}, \texttt{AND}, \texttt{OR}, \texttt{CMP} (Compare),\…
     \item Division flottante (\texttt{FDIV}), Sinus/Cosinus (\texttt{FSIN, FCOS, FSINCOS}),\…
     \item Accès aux périphériques, changement de fréquence du processeur,\…
    \end{itemize}
  \item Le language Assembleur permet de représenter un code binaire:
    \begin{itemize}
      \item Code binaire : 10110000 01100001
      \item Code assembleur : \texttt{movb \$0x61,\%al}\\
            10110000 = \texttt{movb \%al}\\
            01100001 = \texttt{\$0x61}
      \item Action : écrire le nombre 0x61 = 97 dans le registre al.
    \end{itemize}
  \item Chaque processeur a son jeu d'instruction (IS) : 
    \begin{itemize}
      \item x86 / x86\_64
      \item ARM
      \item RISC-V
      \item MIPS
      \item \…
    \end{itemize}
\end{itemize}
\end{frame}

\subsection{Du code source au binaire}

\subsubsection{Le Préprocesseur}
\begin{frame}[fragile]
\frametitle{\insertsubsubsection}
\begin{itemize}
  \item Une étape de transformation \textit{statique} du code source.
  \item Cherche les fichiers inclus par : \texttt{\#include "mon\_header.h"}
  \item \textit{Remplace} la ligne du \texttt{\#include} par le \textit{contenu} du fichier inclus
  \item Insère les macros là où elles sont appelées
\begin{tabular}{p{5cm}p{5cm}}
\begin{lstlisting}[caption=]
#include <stdio.h>
#define HELLO_FR "Bonjour !"
#define CARRE(x) x*x

int main()
{
  printf(HELLO_FR "\n");
  printf("5^2 = %d", CARRE(5));
}
\end{lstlisting} & 
\begin{lstlisting}[caption=]
<contenu de stdio.h recopié ici>



int main()
{
  printf("Bonjour !" "\n");
  printf("5^2 = %d", 5*5);
}
\end{lstlisting}
\end{tabular}
\item On doit indiquer au préprocesseur où chercher les includes :
\begin{itemize}
    \item Par défault : \texttt{.}, \texttt{/usr/include}
    \item Autres exemples : \texttt{./ma\_librairie}, \texttt{/usr/include/gtk-3.0},\…
\end{itemize}
\end{itemize}
\end{frame}

\subsubsection{Les dangers du préprocesseur}
\begin{frame}[fragile]
\frametitle{\insertsubsubsection}
\begin{center}
\begin{tabular}{p{5.5cm} p{5.5cm}}
\begin{lstlisting}[caption=]
#include <stdio.h>
#define OK printf("valeur valide");
#define NOK printf("valeur erronée");

if (is_valid(n))
  OK;
else
  NOK;
\end{lstlisting} & 
\begin{onlyenv}<2->
\begin{lstlisting}[caption=]
<contenu de stdio.h recopié ici>



if (is_valid(n))
  printf("valeur valide");;
else
  printf("valeur erronée");;

\end{lstlisting}
\end{onlyenv} \\
\begin{lstlisting}[caption=]
#define SOMME(x,y) x + y

printf("(10 + 10) / 2 = %d", 
    SOMME(10, 10) / 2);
\end{lstlisting} &
\begin{onlyenv}<3->
\begin{lstlisting}[caption=]


printf("(10 + 10) / 2 = %d", 
    10 + 10 / 2);
\end{lstlisting}
\end{onlyenv}
\onslide<4->{\\
\multicolumn{2}{c}{ \url{https://gcc.gnu.org/onlinedocs/cpp/Macro-Pitfalls.html} }
\\
}

\end{tabular}
\end{center}
\end{frame}

\subsubsection{Le Compilateur}
\begin{frame}
\frametitle{\insertsubsubsection}
\begin{itemize}
  \item Invoquer le préprocesseur, de façon transparente
    \begin{itemize}
      \item Chercher les \texttt{includes}
    \end{itemize}
  \item Optimiser le code
    \begin{itemize}
      \item Boucles vides, regroupement ou réorganisation d'opérations,\…
    \end{itemize}
  \item Générer le code machine
  \item Optimiser le code machine en fonction du processeur cible
    \begin{itemize}
      \item Unités spécifiques (vectorisation, calcul flottant,…)
    \end{itemize}
  \item Ne génère \textit{pas} un exécutable fonctionnel, seulement un fichier objet avec \textit{votre} code
    \begin{itemize}
      \item ne contient pas les fonctions appelées (\texttt{printf}, \texttt{std::*},)
      \item ne "sait" pas où sont les librairies (SDL, GTK, Vulkan,\…)
      \item ne contient pas le code d'initialisation
    \end{itemize}
\end{itemize}

\end{frame}

\subsubsection{Le Linker (éditeur de liens)}
\begin{frame}
\frametitle{\insertsubsubsection}

\begin{itemize}
  \item Regroupe les fichiers objet (fichiers \texttt{*.o} issus de vos \texttt{*.c})
  \item Cherche les librairies (GTK,\…) $\rightarrow$ il faut lui indiquer où les chercher
  \item Résout les symboles (cherche les fonctions appelées d'un objet à l'autre)
    \begin{itemize}
      \item La fonction existe-t-elle ?
      \item Y a-t-il plusieurs définitions ? (conflits)
      \item Enregistrement du lien (\textasciitilde adresse de la fonction)
    \end{itemize}
  \item Détermine les "sections" de l'exécutable
    \begin{description}
      \item[text] section d'instructions
      \item[data/bss] sections de données (chaînes de caractère,…)
      \item[eeprom] section de stockage de données (oui, dans le binaire\…)
    \end{description}
  \item Génère un exécutable fonctionnel
    \begin{itemize}
      \item Rajoute le code d'initialisation (\texttt{\_start})
    \end{itemize}
\end{itemize}

\end{frame}
