\documentclass[8pt,a9paper]{beamer}
\usepackage[french]{babel}
\usepackage[T1]{fontenc}
\usepackage{amsmath}
\usepackage{amsfonts}
\usepackage{amssymb}
\usepackage{listings}
\usepackage{color}
%\usepackage{minted} % Needs --shell-escape and pygmentize
\usepackage{tikz}
\usepackage{colortbl}
\usepackage{array}

\usetheme{Boadilla}
\useoutertheme{Compilation} \setbeamerfont{headline}{size=\small}
%\setbeamertemplate{itemize subitem}[circle]

\newcommand{\titp}[1]{\begin{center}\large{\textbf{#1}}\end{center}}
\newcommand{\image}[3]{\begin{tabular}{c}\includegraphics[#1]{#2}\\ #3 \end{tabular}}

\setlength{\unitlength}{1mm}
\definecolor{vertpale}{rgb}{0.7, 0.9, 0.3}
\definecolor{jaunepale}{rgb}{1, 0.95, 0.41}
\definecolor{rougepale}{rgb}{0.77, 0.3, 0.32}

\lstdefinelanguage
   {Assembler_x64}     % add a "x64" dialect of Assembler
   [x86masm]{Assembler} % based on the "x86masm" dialect
   % with these extra keywords:
   {morekeywords={CDQE,CQO,CMPSQ,CMPXCHG16B,JRCXZ,LODSQ,MOVSXD,MOVB, %
                  POPFQ,PUSHFQ,SCASQ,STOSQ,IRETQ,RDTSCP,SWAPGS, %
                  rax,rdx,rcx,rbx,rsi,rdi,rsp,rbp, %
                  r8,r8d,r8w,r8b,r9,r9d,r9w,r9b, %
                  r10,r10d,r10w,r10b,r11,r11d,r11w,r11b, %
                  r12,r12d,r12w,r12b,r13,r13d,r13w,r13b, %
                  r14,r14d,r14w,r14b,r15,r15d,r15w,r15b}}

\definecolor{mygreen}{rgb}{0,0.6,0}
\definecolor{mygray}{rgb}{0.5,0.5,0.5}
\definecolor{mymauve}{rgb}{0.58,0,0.82}
\lstset{ 
  backgroundcolor=\color{white},   % choose the background color; you must add \usepackage{color} or \usepackage{xcolor}; should come as last argument
  basicstyle=\footnotesize,        % the size of the fonts that are used for the code
  breakatwhitespace=false,         % sets if automatic breaks should only happen at whitespace
  breaklines=true,                 % sets automatic line breaking
  captionpos=b,                    % sets the caption-position to bottom
  commentstyle=\color{mygreen},    % comment style
  deletekeywords={...},            % if you want to delete keywords from the given language
  escapeinside={\%*}{*)},          % if you want to add LaTeX within your code
  extendedchars=true,              % lets you use non-ASCII characters; for 8-bits encodings only, does not work with UTF-8
%  firstnumber=1000,                % start line enumeration with line 1000
  frame=single,	                   % adds a frame around the code
  keepspaces=true,                 % keeps spaces in text, useful for keeping indentation of code (possibly needs columns=flexible)
  keywordstyle=\color{blue},       % keyword style
  language=C,                 % the language of the code
  morekeywords={*,...},            % if you want to add more keywords to the set
%  numbers=left,                    % where to put the line-numbers; possible values are (none, left, right)
  numbersep=5pt,                   % how far the line-numbers are from the code
  numberstyle=\tiny\color{mygray}, % the style that is used for the line-numbers
  rulecolor=\color{black},         % if not set, the frame-color may be changed on line-breaks within not-black text (e.g. comments (green here))
  showspaces=false,                % show spaces everywhere adding particular underscores; it overrides 'showstringspaces'
  showstringspaces=false,          % underline spaces within strings only
  showtabs=false,                  % show tabs within strings adding particular underscores
  stepnumber=2,                    % the step between two line-numbers. If it's 1, each line will be numbered
  stringstyle=\color{mymauve},     % string literal style
  tabsize=2,	                   % sets default tabsize to 2 spaces
  title=\lstname                   % show the filename of files included with \lstinputlisting; also try caption instead of title
}
\newcommand{\…}{\dots}

\title{Introduction à la compilation}
\subtitle{Une petite introduction à la programmation de microcontrôleurs ARM}
\date{8 janvier 2020}
\author{Félix Piédallu, Club Robotronik}

\begin{document}

\begin{frame}
  \titlepage
\end{frame}

\begin{frame}
  \frametitle{\textbf{Sommaire}}
  \tableofcontents
\end{frame}


\section{Qu'est-ce que la compilation ?}

\subsection{Comment qu'on exécute du code ?}
\begin{frame}[fragile]
\frametitle{\insertsubsection}
\begin{block}{Du code source au résultat}

  \begin{tikzpicture}[main node/.style={rectangle,draw,minimum size=0.5cm,align=left},]
        
    \node[main node, label=\textbf{Source :} \texttt{main.c}] (1) {
      \texttt{int main()\{}\\
      \texttt{~~~~printf("Hello");}\\
      \texttt{~~~~\…}\\
      \texttt{\}}
    };
    \node[main node] (2) [right = 1cm of 1] {Compilation};
    \node[main node, label=Code objet] (3) [right = 1cm of 2] {
      \texttt{01001001}\\
      \texttt{10010100}\\
      \texttt{11000011}\\
      \texttt{10011011}
    };
    \node[main node, label=Résultat de l'exécution] (4) [right = 1cm of 3] {
      \texttt{Hello}\\
      \texttt{The result is 42}
    };

    \draw[->] (1) edge (2) (2) edge (3) (3) edge (4);
  \end{tikzpicture}
\end{block}

\begin{block}{Les étapes de la compilation}
  \begin{tikzpicture}[main node/.style={rectangle,draw,minimum size=0cm,align=left},]
        
    \node[main node, label=Fichier \texttt{.c}] (1) {Code Source};
    \node[main node, label=Préprocesseur] (2) [right = 0.4cm of 1] {
      Code reformaté:\\
      Recherche des includes\\
      Résolution des macros
    };
    \node[main node, label=Compilateur] (3) [right = 0.4cm of 2] {
      Code objet\\
      Optimisation
    };
    \node[main node, label=Linker (Éditeur de liens)] (4) [right = 0.3cm of 3] {
      Recherche des librairies\\
      Résolution des symboles\\
      Génération de l'exécutable/librarie
    };

    \draw[->] (1) edge (2) (2) edge (3) (3) edge (4);
  \end{tikzpicture}
\end{block}
\end{frame}


\subsection{Parler avec le processeur : le jeu d'instructions}

\begin{frame}[fragile]
\frametitle{\insertsubsection}
\begin{itemize}
  \item Ensemble des instructions exécutables par un processeur
    \begin{itemize}
    \item \texttt{ADD}, \texttt{AND}, \texttt{OR}, \texttt{CMP} (Compare),\…
     \item Division flottante (\texttt{FDIV}), Sinus/Cosinus (\texttt{FSIN, FCOS, FSINCOS}),\…
     \item Accès aux périphériques, changement de fréquence du processeur,\…
    \end{itemize}
  \item Le language Assembleur permet de représenter un code binaire:
    \begin{itemize}
      \item Code binaire : 10110000 01100001
      \item Code assembleur : \texttt{movb \$0x61,\%al}\\
            10110000 = \texttt{movb \%al}\\
            01100001 = \texttt{\$0x61}
      \item Action : écrire le nombre 0x61 = 97 dans le registre al.
    \end{itemize}
\end{itemize}
\end{frame}

\subsection{Du code source au binaire}

\subsubsection{Le Préprocesseur}
\begin{frame}[fragile]
\frametitle{\insertsubsubsection}
\begin{itemize}
  \item Une étape de transformation \textit{statique} du code source.
  \item Cherche les fichiers inclus par : \texttt{\#include "mon\_header.h"}
  \item \textbf{Remplace} la ligne du \texttt{\#include} par le \textit{contenu} du fichier inclus
  \item Insère les macros là où elles sont appelées
\begin{tabular}{p{5cm}p{5cm}}
\begin{lstlisting}
#include <stdio.h>
#define HELLO_FR "Bonjour !"
#define CARRE(x) x*x

int main()
{
  printf(HELLO_FR "\n");
  printf("5^2 = %d", CARRE(5));
}
\end{lstlisting} & 
\begin{lstlisting}
<contenu de stdio.h recopié ici>



int main()
{
  printf("Bonjour !" "\n");
  printf("5^2 = %d", 5*5);
}
\end{lstlisting}
\end{tabular}
\end{itemize}
\end{frame}

\subsubsection{Les dangers du préprocesseur}
\begin{frame}[fragile]
\frametitle{\insertsubsubsection}
\begin{center}
\begin{tabular}{p{5.5cm} p{5.5cm}}
\begin{lstlisting}[caption=]
#include <stdio.h>
#define OK printf("valeur valide");
#define NOK printf("valeur erronée");

if (is_valid(n))
  OK;
else
  NOK;
\end{lstlisting} & 
\begin{onlyenv}<2->
\begin{lstlisting}[caption=]
<contenu de stdio.h recopié ici>



if (is_valid(n))
  printf("valeur valide");;
else
  printf("valeur erronée");;

\end{lstlisting}
\end{onlyenv} \\
\begin{lstlisting}[caption=]
#define CIRCONFERENCE(x,y) x + x + y + y

printf("demi-circonférence: %d", 
    CIRCONFERENCE(10, 10) / 2);
\end{lstlisting} &
\begin{onlyenv}<3->
\begin{lstlisting}[caption=]
#define CIRCONFERENCE(x,y) x + x + y + y

printf("demi-circonférence: %d", 
    x + x + y + y / 2);
\end{lstlisting}
\end{onlyenv}
\onslide<4->{\\
\multicolumn{2}{c}{ \url{https://gcc.gnu.org/onlinedocs/cpp/Macro-Pitfalls.html} }
\\
}

\end{tabular}
\end{center}
\end{frame}

\subsubsection{Le Compilateur}
\begin{frame}
\frametitle{\insertsubsubsection}
\begin{itemize}
  \item Invoquer le compilateur de façon transparente
    \begin{itemize}
      \item Chercher les \texttt{includes}
    \end{itemize}
  \item Optimiser le code
    \begin{itemize}
      \item Boucles vides, regroupement ou réorganisation d'opérations,\…
    \end{itemize}
  \item Générer le code machine
  \item Optimiser le code machine en fonction du processeur cible
    \begin{itemize}
      \item Unités spécifiques (vectorisation, calcul flottant,…)
    \end{itemize}
  \item Ne génère \textit{pas} un exécutable fonctionnel, seulement un fichier objet avec \textit{votre} code
    \begin{itemize}
      \item ne contient pas les fonctions appelées (\texttt{printf}, \texttt{std::*},)
      \item ne "sait" pas où sont les librairies (SDL, GTK, Vulkan,\…)
      \item ne contient pas le code d'initialisation
    \end{itemize}
\end{itemize}

\end{frame}

\subsubsection{Le Linker (éditeur de liens)}
\begin{frame}
\frametitle{\insertsubsubsection}

\begin{itemize}
  \item Regroupe les fichiers objet (fichiers \texttt{*.o} issus de vos \texttt{*.c})
  \item Cherche les librairies (GTK,\…)
  \item Résout les symboles (cherche les fonctions appelées d'un objet à l'autre)
    \begin{itemize}
      \item La fonction existe-t-elle ?
      \item Y a-t-il plusieurs définitions ? (conflits)
      \item Enregistrement du lien (\textasciitilde adresse de la fonction)
    \end{itemize}
  \item Détermine les "sections" de l'exécutable
    \begin{description}
      \item[text] section d'instructions
      \item[data/bss] sections de données (chaînes de caractère,…)
      \item[eeprom] section de stockage de données (oui, dans le binaire\…)
    \end{description}
  \item Génère un exécutable fonctionnel
    \begin{itemize}
      \item Rajoute le code d'initialisation (\texttt{\_start})
    \end{itemize}
\end{itemize}

\end{frame}



\section{Outils de compilation}
\begin{frame} auie \end{frame}
\subsection{Script de compilation}
\begin{frame} auie \end{frame}
\subsection{Make}
\begin{frame} auie \end{frame}
\subsection{Générateurs de Makefiles}
\begin{frame} auie \end{frame}
\subsection{Meson}
\begin{frame} auie \end{frame}


\end{document}
