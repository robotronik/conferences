
\section{Organiser la compilation}

\begin{frame}{Un joyeux bordel à organiser}

  \textbf{\large Il faut gérer tout ça}
  \par
  \begin{itemize}
    \item Chaque outil a ses spécificités
    \item Chaque outil doit être exécuté avec les options (arguments) adéquats
  \end{itemize}
  \bigskip
  \textbf{\large On veut :}
  \begin{itemize}
    \item Compiler facilement
    \item Une compilation reproductible
  \end{itemize}
  \bigskip
  \textbf{\large Il nous faut alors un script qui :}
  \begin{itemize}
    \item Liste les arguments de chaque outil
    \item Exécute les étapes dans le bon ordre
    \item Exécute uniquement les étapes nécessaires :
      \begin{itemize}
        \item ne recompile que si c'est nécessaire
        \item détermine automatiquement les fichiers à recompiler
      \end{itemize}
    \item Parallélise les tâche sur le CPU
    \item Doit être simple à écrire et à lire
  \end{itemize}

\end{frame}

\subsection{Script de compilation}
\begin{frame}[fragile]
\frametitle{\insertsubsection}

On pourrait écrire un script de compilation\…
\begin{lstlisting}[caption=]
#!/bin/bash

# Mes sources : main.c, audio_output.c, audio_output.h, ogg.h

gcc -Wall -Wextra -I/usr/include/ogg -o main.o -c main.c 

gcc -Wextra -I/usr/include/libsoundio -o audio_outupt.o -c audio_output.c

gcc -lsoundio -logg main.o audio_output.o -o audio_notification

\end{lstlisting}
\onslide<2->{
\medskip
Mais :
\begin{itemize}
  \item Recompile tout, même si ce n'est pas nécessaire
  \item Ne fait qu'une étape à la fois, alors que j'ai un processeur 12 coeurs
  \item Ne détecte pas les changements dans les headers
  \item Les arguments de compilation ne sont pas uniformes
  \item Je me suis planté sur le nom d'un fichier
\end{itemize}
}
\end{frame}

\subsection{Make}
\begin{frame}[fragile]
\frametitle{\insertsubsection}

\begin{lstlisting}[language=make, caption=Un Makefile classique]
FILES_SRC = \
    main.c \
    give_result.c \
    robotronik_is_life.c

FILES_HEADER = \
    give_result.h \
    robotronik.h

FILES_OBJ = $(FILES_SRC:.c=.o)

result: executable
    executable --give-me-result

executable: $(FILES_OBJ)
    link_those_files $^ --output=$@

%.o: %.c $(FILES_HEADER)
    compile_dat_file $< -o $@
\end{lstlisting}
\end{frame}

\subsubsection{La syntaxe d'un Makefile}

\begin{frame}[fragile]
\frametitle{\insertsubsubsection}
\begin{lstlisting}[language=make, caption=]
# Les lignes commençant par # sont ignorées.
# On peut déclarer et utiliser des variables:
Ma_Variable = tomate
Ma_Liste = pomme poire abricot $(Ma_Variable)

# Une *règle* est décrite par la cible, les dépendances et la recette

cible: file1.c file2.c
    # La recette est un script bash sur plusieurs lignes. On y écrit ce qu'on veut.
    echo Ma liste contient $(Ma_Liste).
    gcc $^ -o $@
    @echo Fichier compilé !

# Une cible n'est pas forcément un fichier : ça peut être une "commande"
run: cible
    ./cible --run
\end{lstlisting}
\end{frame}
\begin{frame}[fragile]
\frametitle{\insertsubsubsection}
La syntaxe d'une \textit{règle} :
\begin{lstlisting}[language=make, caption=]
cible: dépendance1 dépendance2
    recette
    @recette muette
    @echo la première dépendance est $<
    compile --output=$@ --deps $^
\end{lstlisting}
\medskip
On peut rendre une règle muette en rajoutant un \texttt{@} au début de la ligne.

\bigskip
Des variables magiques :
\begin{description}
  \item[\texttt{\textdollar@}] La cible
  \item[\texttt{\textdollar<}] La première dépendance
  \item[\texttt{\textdollar\textasciicircum}] Toutes les dépendances
\end{description}

\end{frame}

\begin{frame}[fragile]
\frametitle{\insertsubsubsection}

On peut écrire des règles génériques, par exemple si plusieurs fichiers \texttt{*.o} doivent 
être compilés \textit{depuis} leur \texttt{*.c} correspondant :

\begin{lstlisting}
%.o: %.c
    compile_dat_file $< -o $@
\end{lstlisting}
\medskip
Par exemple si une autre règle est :
\begin{lstlisting}
sudoku: sudoku.o auto_solver.o
    link_those_files $< -o $@
\end{lstlisting}

Make saura alors qu'il faut compiler \texttt{sudoku.c} et \texttt{auto\_solver.c}.

\end{frame}

\begin{frame}[fragile]
\frametitle{\insertsubsubsection}

On utilise un Makefile à l'aide de la commande \texttt{make}.

\begin{lstlisting}[language=bash]
$ make run
# La recette est un script bash sur plusieurs lignes. On y écrit ce qu'on veut.
echo Ma liste contient pomme poire abricot tomate.
Ma liste contient pomme poire abricot tomate.
gcc file1.c file2.c -o cible
Fichier compilé !
./cible --run
Hello world !
\end{lstlisting}



Make fait le strict minimum : pas besoin de recompiler un fichier si les sources ne changent pas.

\begin{lstlisting}[language=bash]

$ make cible
make: « cible » est à jour.
\end{lstlisting}
\end{frame}

\begin{frame}{C'est compliqué tout ça\…}
\sout{Ta gueule et code} Mais non, code, on est bien !

\begin{itemize}
  \item La syntaxe "de base" est finalement assez simple
  \item Vous en apprendrez plus au fil du temps. Il faut juste déjà que vous sachiez lire et utiliser un Makefile.
  \item Vous verrez des Makefile \textit{partout}
  \item Ça simplifie réellement le boulot : \texttt{make flash} et tout est automatique !
\end{itemize}

\end{frame}

\begin{frame}[fragile]{Petits bonus pour férus d'automatisation}
\begin{itemize}
\item Changer automatiquement l'extension des fichiers d'une liste:
\begin{lstlisting}
SOURCES = main.c pid.c asservissement.c hardware.c
OBJECTS = $(SOURCES:.c=.o)
\end{lstlisting}

\item Automatisation de la détection des includes:
\begin{lstlisting}[title=Makefile]
CFLAGS += -MMD
-include $(OBJECTS:.o=.d)
\end{lstlisting}
\medskip
\begin{lstlisting}[title=file1.d]
file1.o: file1.c file2.h

\end{lstlisting}
\end{itemize}
\end{frame}

\begin{frame}[fragile]{Petits bonus pour férus d'automatisation}
\begin{itemize}
\item Trouver automatiquement les dépendances grâce à Pkg-config
\begin{lstlisting}[title=\texttt{/usr/lib/pkgconfig/libpng.pc}]
prefix=/usr
exec_prefix=${prefix}
libdir=${exec_prefix}/lib
includedir=${prefix}/include/libpng16

Name: libpng
Description: Loads and saves PNG files
Version: 1.6.37
Requires: zlib
Libs: -L${libdir} -lpng16
Libs.private: -lm -lz -lm 
Cflags: -I${includedir}
\end{lstlisting}

\begin{lstlisting}[title=Code du makefile]
CFLAGS += $(pkg-config --cflags libpng)
# Résoudra en : -I/usr/include/libpng16

LDFLAGS += $(pkg-config --libs libpng)
# Résoudra en : -lpng16 -lz
\end{lstlisting}
\end{itemize}
\end{frame}

\section{Générateurs de Makefiles}
\begin{frame}[fragile]{C'est compliqué tout ça\…}
Ouai, bon, d'accord, c'est pas la syntaxe la plus propre du monde.

\… Mais il existe des outils plus "propres" qui génèrent des Makefiles !
\begin{lstlisting}
project('mYBigGesTPrOJecTeVeR',
  'cpp',
  default_options: [
    'buildtype=release',
    'warning_level=3',
  ],
)

ogg = dependency('ogg', version: '>=0.4.0')
mp3 = dependency('mp3', version: '>=1.0', required: false)

sources = [ 'converter_ogg.c' ]
if mp3.found()
  sources += 'converter_mp3.c'
endif

converter = executable('converter',
  sources,
  dependencies: [ ogg, mp3 ],
  install: true,
)
  
\end{lstlisting}
\end{frame}

\begin{frame}{Quelques générateurs de scripts}
\begin{itemize}
  \item Autotools \onslide<2>{{\footnotesize {\usebeamercolor[fg]{structure} \textbf{(bloated, lent, non portable,\…)}}}}
  \item CMake \onslide<2>{{\footnotesize {\usebeamercolor[fg]{structure} (Langage  string-based, très souple, très portable)}}}
  \item qmake \onslide<2>{{\footnotesize {\usebeamercolor[fg]{structure} (Orienté Qt / QtCreator)}}}
  \item Scons \onslide<2>{{\footnotesize {\usebeamercolor[fg]{structure} (\textit{Lent})}}}
  \item Bazel \onslide<2>{{\footnotesize {\usebeamercolor[fg]{structure} (Développé par Google, pas très libre, java)}}}
  \item Meson \onslide<2>{{\footnotesize {\usebeamercolor[fg]{structure} (\textit{Très rapide})}}}
\end{itemize}
\end{frame}

\begin{frame} auie \end{frame}
\subsection{Meson}
\begin{frame} auie \end{frame}
