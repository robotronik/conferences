\chapter{Linux c'est quoi ?}

Linux ou plus précisément GNU/Linux est un système d'exploitation basé sur un noyau UNIX. À la différence de Windows, qui est basé sur un noyau DOS, Linux est open source, c'est à dire que tout le monde peut avoir accès aux code sources et les modifiés à sa guise. La seule règle étant de laisser son œuvre libre et disponible à tous.

Il a été crée par Linus Torvalds en 1991 et est un veteur important de populrsation du mouvement open source. En effet, les codes composants le noyeau (kernel) sont disponibles en ligne sur la page github de Linus Torvalds (\url{https://github.com/torvalds/linux}). De ce fait, tous le monde peut réadapter à sa sauce le noyeau pour en faire une variante.

\section{Linux ça se retrouve où ?}
Développé à la base pour des ordinateurs personnel (PC) il se retrouve maintenant sur une grande partie du matériel informatique, allant des pc jusqu'au super-ordinateurs en passant par les téléphones portables. Et oui, Android le système d'exploitation équipant 80\% des des smartphones possède un noyeau linux.

\section{Les distributions}
À la différence de Windows ou MacOS, il existe différentes "versions" de linux. On appelle ces "versions" distributions.


Il en existe toute une floppées qui ont chacune leurs spécificités. Dans les plus connues, on retrouve :
\begin{itemize}
	\item Debian
	\item Ubuntu
	\item Linux Mint
	\item Arch Linux
	\item Manjaro Linux
\end{itemize}


