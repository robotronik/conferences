\documentclass{beamer}
\usepackage{tikz}
\usepackage[french]{babel}
\usepackage{caption}

\definecolor{blackRobotronik}{RGB}{19,19,19}

% setting some colors for the theme
\setbeamercolor{palette primary}{fg=blackRobotronik,bg=white}
\setbeamercolor{palette secondary}{fg=blackRobotronik,bg=white}
\setbeamercolor{structure}{fg=white,bg=blackRobotronik}
\setbeamercolor{title in head/foot}{fg=white,bg=blackRobotronik}
\setbeamercolor{date in head/foot}{fg=white,bg=blackRobotronik}

% definition of the footline template
\defbeamertemplate*{footline}{Robotronik}{%
  \leavevmode%
  \hbox{\begin{beamercolorbox}[wd=.5\paperwidth,ht=2.5ex,dp=1.125ex,leftskip=.3cm,rightskip=.3cm]{title in head/foot}%
    \makebox[2em][l]{{\usebeamerfont{title in head/foot}\textcolor{blackRobotronik}{\insertframenumber}}}%
    {\usebeamercolor{title in head/foot}\insertframenumber}
  \end{beamercolorbox}%
  \begin{beamercolorbox}[wd=.2\paperwidth,ht=2.5ex,dp=1.125ex,leftskip=.3cm,rightskip=.3cm]{date in head/foot}%
    \usebeamerfont{date in head/foot}\insertshorttitle%
  \end{beamercolorbox}%
  \begin{beamercolorbox}[wd=.3\paperwidth,ht=2.5ex,dp=1.125ex,leftskip=.3cm,rightskip=.3cm]{title in head/foot}%
  \insertshortdate
    %\includegraphics[width=.2\paperwidth,height=2.5ex,keepaspectratio]{gemalto}\hspace*{2em}%
  \end{beamercolorbox}}%
  \vskip0pt%
}

% definition of the title page template
\defbeamertemplate*{title page}{Robotronik}[1][]
{%
  \begin{tikzpicture}[remember picture,overlay]
  \filldraw[blackRobotronik]
    (current page.north west) --
    ([yshift=-2cm]current page.north west) --
    ([xshift=-2cm,yshift=-2cm]current page.north east) {[rounded corners=15pt]--
    ([xshift=-2cm,yshift=3cm]current page.south east)} --
    ([yshift=3cm]current page.south west) --
    (current page.south west) --
    (current page.south east) --
    (current page.north east) -- cycle
    ;
  \node[text=blackRobotronik,anchor=south west,font=\sffamily\LARGE,text width=.55\paperwidth] 
  at ([xshift=10pt,yshift=-1cm]current page.west)
  (title)
  {\raggedright\inserttitle};   
  \node[anchor=west]
  at (title.east)
  {\includegraphics[height=2.25cm]{Images/logo.png}};
  \node[text=white,font=\large\sffamily,anchor=south west]
  at ([xshift=30pt,yshift=1cm]current page.south west)
  (date)
  {\insertdate};
  \node[text=white,font=\large\sffamily,anchor=south west]
  at ([yshift=5pt]date.north west)
  (author)
  {\insertauthor};
  \end{tikzpicture}%
}

% remove navigation symbols
\setbeamertemplate{navigation symbols}{}

\setbeamertemplate{itemize item}[circle]
\setbeamercolor{itemize item}{fg=blackRobotronik}
\setbeamertemplate{itemize subitem}[square]
\setbeamercolor{itemize subitem}{fg=blackRobotronik}
\setbeamertemplate{itemize subsubitem}[cicle]
\setbeamercolor{itemize subsubitem}{fg=blackRobotronik}

\newenvironment{slide}[1]{
  \begin{frame}[environment=slide]
    \frametitle{\textbf{\insertsection}  : #1}}
{\end{frame}}

\setbeamercolor{section in toc}{fg = blackRobotronik}
\setbeamercolor{section in toc shaded}{fg = blackRobotronik}
\setbeamertemplate{section in toc}[square]
\setbeamerfont{section number projected}{size=\large}
\setbeamercolor{section number projected}{bg=blackRobotronik,fg=white}

\setbeamercolor{caption name}{fg=blackRobotronik}
\setbeamertemplate{caption}[numbered]
\setbeamercolor{caption number}{fg=blackRobotronik}

\title{Linux, les bases}
\author{Nicolas Cubric}
\date{13/11/2019 - M001}

\begin{document}

\begin{frame}
\maketitle
\end{frame}

\begin{frame}
  \frametitle{Sommaire}
  \tableofcontents
\end{frame}

\section{Introduction}
\subsection{Linux c'est quoi ?}

\begin{slide}{Linux c'est quoi ?}
  \begin{center}
    \includegraphics[scale=0.25]{Images/GNU_TUX.png}
    \captionof{figure}{Logo de GNU/Linux}
  \end{center}
\end{slide}

\begin{slide}{Linux c'est quoi ?}
  \begin{itemize}
    \item Un système d'exploitation
      \begin{itemize}
        \item crée en 1991 par Linus Torvalds.
        \item écrit en C et en assembleur
      \end{itemize}
    \item Utilisé de partout
    \begin{itemize}
      \item Ordinateurs personnel
      \item Téléphones portables/Tablettes
      \item Super-ordinateurs
    \end{itemize}
  \end{itemize}
\end{slide}

\subsection{Les distributions}

\begin{slide}{Les distributions}
  \begin{center}
    \includegraphics[scale=0.5]{Images/distro_mosaic.png}
    \captionof{figure}{Logo de différentes distributions}
  \end{center}
\end{slide}

\section{L'arborescence}

\begin{slide}{}
  \begin{center}
    \includegraphics[scale=0.3]{Images/arborescence.png}
    \captionof{figure}{Arborescence dans Linux}
  \end{center}
\end{slide}

\begin{slide}{/}
  \begin{itemize}
  \item C'est le dossier à la racine.
  \item Son petit nom c'est root (comme racine en anglais)
  \item De ce dossier pars tout l'arborescence du systéme
  \item Nous allons voir les principals dossiers.
  \end{itemize}
\end{slide}

\begin{slide}{/home}
  \begin{itemize}
  \item C'est la maison !!!
  \item Ici on retrouve toutes les données des utilisateurs
  \item Chaques utilisateurs à son dossier attitré
  \end{itemize}
\end{slide}

\section{Quelques commandes de bases}

\subsection{Le terminal}
\begin{slide}{Ouvrir un terminal}
  \begin{itemize}
  \item Dans la majorité des interfaces graphique : ctr+alt+t
  \item Sinon aller chercher dans la liste des applications
  \end{itemize}
\end{slide}

\subsection{man}

\begin{slide}{man}
  \begin{itemize}
  \item man pour manual
  \item Utile quand on cherche des informations sur une commande
  \end{itemize}
  \begin{center}
    \large{Exemple d'utilisation :\\
      \textbf{À vos pcs !!!}}
  \end{center}
\end{slide}


\end{document}