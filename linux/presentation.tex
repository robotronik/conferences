\documentclass[13pt]{beamer}
\usepackage{tikz}
\usepackage[french]{babel}
\usepackage{url}
\usepackage{caption}
\usepackage{xcolor}
\usepackage{setspace}
\usepackage{listings}
\lstset{basicstyle=\ttfamily,
  showstringspaces=false,
  commentstyle=\color{red},
  keywordstyle=\color{blue}
}

%\onehalfspacing

\definecolor{blackRobotronik}{RGB}{19,19,19}

% setting some colors for the theme
\setbeamercolor{palette primary}{fg=blackRobotronik,bg=white}
\setbeamercolor{palette secondary}{fg=blackRobotronik,bg=white}
\setbeamercolor{structure}{fg=white,bg=blackRobotronik}
\setbeamercolor{title in head/foot}{fg=white,bg=blackRobotronik}
\setbeamercolor{date in head/foot}{fg=white,bg=blackRobotronik}

% definition of the footline template
\defbeamertemplate*{footline}{Robotronik}{%
  \leavevmode%
  \hbox{\begin{beamercolorbox}[wd=.5\paperwidth,ht=2.5ex,dp=1.125ex,leftskip=.3cm,rightskip=.3cm]{title in head/foot}%
    \makebox[2em][l]{{\usebeamerfont{title in head/foot}\textcolor{blackRobotronik}{\insertframenumber}}}%
    {\usebeamercolor{title in head/foot}\insertframenumber}
  \end{beamercolorbox}%
  \begin{beamercolorbox}[wd=.2\paperwidth,ht=2.5ex,dp=1.125ex,leftskip=.3cm,rightskip=.3cm]{date in head/foot}%
    \usebeamerfont{date in head/foot}\insertshorttitle%
  \end{beamercolorbox}%
  \begin{beamercolorbox}[wd=.3\paperwidth,ht=2.5ex,dp=1.125ex,leftskip=.3cm,rightskip=.3cm]{title in head/foot}%
  \insertshortdate
    %\includegraphics[width=.2\paperwidth,height=2.5ex,keepaspectratio]{gemalto}\hspace*{2em}%
  \end{beamercolorbox}}%
  \vskip0pt%
}

% definition of the title page template
\defbeamertemplate*{title page}{Robotronik}[1][]
{%
  \begin{tikzpicture}[remember picture,overlay]
  \filldraw[blackRobotronik]
    (current page.north west) --
    ([yshift=-2cm]current page.north west) --
    ([xshift=-2cm,yshift=-2cm]current page.north east) {[rounded corners=15pt]--
    ([xshift=-2cm,yshift=3cm]current page.south east)} --
    ([yshift=3cm]current page.south west) --
    (current page.south west) --
    (current page.south east) --
    (current page.north east) -- cycle
    ;
  \node[text=blackRobotronik,anchor=south west,font=\sffamily\LARGE,text width=.55\paperwidth] 
  at ([xshift=10pt,yshift=-1cm]current page.west)
  (title)
  {\raggedright\inserttitle};   
  \node[anchor=west]
  at (title.east)
  {\includegraphics[height=2.25cm]{Images/logo.png}};
  \node[text=white,font=\large\sffamily,anchor=south west]
  at ([xshift=30pt,yshift=1cm]current page.south west)
  (date)
  {\insertdate};
  \node[text=white,font=\large\sffamily,anchor=south west]
  at ([yshift=5pt]date.north west)
  (author)
  {\insertauthor};
  \end{tikzpicture}%
}

% remove navigation symbols
\setbeamertemplate{navigation symbols}{}

\setbeamertemplate{itemize item}[circle]
\setbeamercolor{itemize item}{fg=blackRobotronik}
\setbeamertemplate{itemize subitem}[square]
\setbeamercolor{itemize subitem}{fg=blackRobotronik}
\setbeamertemplate{itemize subsubitem}[cicle]
\setbeamercolor{itemize subsubitem}{fg=blackRobotronik}

\newenvironment{slide}[1]{
  \begin{frame}[environment=slide]
    \frametitle{\textbf{\insertsection}  : #1}}
{\end{frame}}

\setbeamercolor{section in toc}{fg = blackRobotronik}
\setbeamercolor{section in toc shaded}{fg = blackRobotronik}
\setbeamertemplate{section in toc}[square]
\setbeamerfont{section number projected}{size=\large}
\setbeamercolor{section number projected}{bg=blackRobotronik,fg=white}

\setbeamercolor{caption name}{fg=blackRobotronik}
\setbeamertemplate{caption}[numbered]
\setbeamercolor{caption number}{fg=blackRobotronik}

\title{Linux, les bases}
\author{Nicolas Cubric}
\date{13/11/2019 - M001}

\begin{document}

\begin{frame}
\maketitle
\end{frame}

\begin{frame}
  \frametitle{\textbf{Sommaire}}
  \tableofcontents
\end{frame}

\section{Introduction}
\subsection{Linux c'est quoi ?}

\begin{slide}{Linux c'est quoi ?}
  \begin{tabular}{cl}
    \begin{tabular}{c}
      \includegraphics[width=0.3\textwidth]{Images/tux.jpg}
    \end{tabular}
    \begin{tabular}{l}
      \parbox{0.7\textwidth}{
      \begin{itemize}
      \item Un système d'exploitation
        \begin{itemize}
        \item créé en 1991 par Linus Torvalds.
        \item écrit en C et en assembleur
        \end{itemize}
      \item Utilisé de partout
        \begin{itemize}
        \item Ordinateurs personnels
        \item Téléphones portables/Tablettes
        \item Super-ordinateurs
        \end{itemize}
      \end{itemize}
      }
    \end{tabular}
  \end{tabular}
\end{slide}

\subsection{Les distributions}

\begin{slide}{Les distributions}
  \begin{center}
    \includegraphics[scale=0.5]{Images/distro_mosaic.png}
    \captionof{figure}{Logo de différentes distributions}
  \end{center}
\end{slide}

\section{L'arborescence}

\begin{slide}{}
  \begin{center}
    \includegraphics[scale=0.3]{Images/arborescence.png}
    \captionof{figure}{Arborescence dans Linux}
  \end{center}
\end{slide}

\subsection{Retrouver son chemin}

\begin{slide}{Retrouver son chemin}
  Il existe deux types de chemins dans Linux :
  \begin{itemize}
  \item Les chemins \textit{absolus}
    \begin{itemize}
    \item On exprime le chemin en partant de la racine
    \item Par exemple le chemin pour mon home : /home/nikola
    \end{itemize}
  \item Les chemins \textit{relatifs}
    \begin{itemize}
    \item On exprime le chemin relatif au dossier courant.
    \item Par exemple le chemin pour mon home : \textasciitilde
    \end{itemize}
  \end{itemize}
\end{slide}

\begin{slide}{Chemin relatif ?}
  Comme précisé dans la slide précédente, il est possible d'exprimer un chemin par
  rapport au dossier dans lequel on se situe.\\

  Si l'on affiche les fichiers cachés on peut remarquer deux dossiers présents
  peu importe où l'on se situe dans l'arborescence :
  \begin{itemize}
  \item . qui est le dossier courant
  \item .. qui est le dossier parent
  \end{itemize}
\end{slide}

\section{Quelques commandes de base}

\subsection{Le terminal}
\begin{slide}{Le terminal}
  \begin{center}
    \includegraphics[scale=0.4]{Images/terminal.png}
  \end{center}
  \begin{itemize}
  \item Dans la majorité des interfaces graphique : ctr+alt+t
  \item Sinon aller chercher dans la liste des applications
  \end{itemize}
\end{slide}

\begin{slide}{Partie pratique}
  \begin{center}
    \LARGE{Exemple d'utilisation :\\
      \textbf{À vos pcs !!!}}
  \end{center}
\end{slide}


\section{Conclusion}

\begin{frame}
  \frametitle{\textbf{Conclusion}}
  Merci de votre attenion.\\
  \vspace*{1cm}
  Les slides sont disponibles sur le github de Robotronik : \url{github.com/robotronik/conferences}
\end{frame}


\end{document}